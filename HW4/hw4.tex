\documentclass{article}

\usepackage[margin=1in]{geometry}
\usepackage{amsmath}
\usepackage{amssymb}

\begin{document}

\title{M 476 - Homework 4}
\author{Nathan Stouffer}

\maketitle
\newpage

% optional
%\section*{Problem 1}

\section*{Problem 2}
Problem: For which
$n \geq 1$ is $SO(n)$
path-connected? Justify your answer. \\\\
% what about n = 1?
Solution: We begin by defining
$SO(n)$:
$$SO(n) := \{ A_{n \times n \text{ matrix}} \mid A^T A = I_{n \times n} \text{ and } det(A) > 0 \}$$
The set
$SO(n)$
can also be thought of as the set of all rotations of
$\mathbb{R}^n$
about the origin. \\\\
We first show that
$n = 1$ is path connected.
$SO(1)$
has only one element: 
$[1]$.
Therefore,
$SO(1)$
is path connected. \\\\
We now shows that, for 
$n \geq 2, \text{ } SO(n)$
is path-connected. Let 
$ V = \{ \vec{v}_1 , \vec{v}_2, ..., \vec{v}_n \}$
be an orthonormal basis (that respects orientation) for 
$\mathbb{R}^n$.
The basis
$V$
must be an element of
$SO(n)$
since the elements of 
$V$
could be the columns of a
$n \times n$
matrix that is both orthonormal and whose determinant is greater than 0 (since 
$V$
respects orientation).
Since
$V$
is an arbitrary element of 
$SO(n)$,
if we can find a continuous map from 
$V$
to another arbitrary element
$E = \{ \vec{e}_1 , \vec{e}_2, ..., \vec{e}_n \}$
then we will know that
$SO(n)$
is path-connected. \\\\
Take an arbitrary natural number 
$k \geq 2$
for
$SO(k)$.
Then, 
$V = \{ \vec{v}_1, \vec{v}_2, ... \vec{v}_k \}$
and
$E = \{ \vec{e}_1, \vec{e}_2, ... \vec{e}_k \}$.
We will prove that
$V$
and
$E$
are connected by inductively rotating each element of 
$V$
to match the position of
$E$. \\\\
We begin with the base case of
$\vec{v}_1$. We will now construct a path from
$\vec{v}_1$
to
$\vec{e}_1$.
Let
$\vec{p}_1$
be a unit vector such that
$\vec{p}_1 \perp \vec{e}_1$
and
$v_1 \in span\{\vec{p}_1, \vec{e}_1\}$.
Then,
$\exists t_1 \in [0, 2 \pi] \text{ such that } \vec{v}_1 = \cos (t) * \vec{e}_1 + \sin (t) * \vec{p}_1$.
Then,
$[0, t_1] \to SO(k)$
is defined by
$t_1 \mapsto (\vec{v}_1, \vec{v}_2, ..., \vec{v}_k)$
and
$0 \mapsto (\vec{e}_1, \vec{v\prime}_2, ..., \vec{v\prime}_k)$.
Since
$\cos (t) \vec{e}_1 + \sin (t) \vec{p}_1$
is made up of continuous functions, the arbitrary elements
$\vec{v}_1$
and
$\vec{e}_1$
must be path-connected. \\\\
Our inductive assumption is as follows: for an arbitrary
$s \geq 2$,
let
$\vec{v}_s \in V$
and assume
$\forall \vec{v}_t \text{ such that } t \leq s$
that
$\vec{v}_t = \vec{e}_t$. \\\\
We will now show that there exists a path from 
$\vec{v}_{s+1}$
to
$\vec{e}_{s+1}$.
Let
$V \prime \in SO(k)$
be such a vector as described in the inductive assumption. Then, 
$V \prime = \{ \vec{e}_1, \vec{e}_2, ... , \vec{e}_s, \vec{v\prime}_{s+1}, ... \vec{v\prime}_k \}$.
Now let
$\vec{p\prime}$
be a vector perpendicular to
$\vec{e}_{s+1}$
such that 
$\vec{v}_{s+1} \in span\{\vec{p\prime}, \vec{e}_{s+1}\}$.
We know that such a vector
$p\prime$
exists because of Graham-Schmitt orthogonolization. Then
$\exists t_2 \in [0, 2 \pi ] \text{ such that } \vec{v}_{s+1} = cos(t) * \vec{e}_{s+1} + sin(t) * \vec{p\prime}$.
Now, we hold all vectors
$\vec{v\prime}_r \in V^\prime$
such that
$\vec{v\prime}_r = \vec{e}_r$
constant and rotate all remaining vectors in
$V^\prime$ according to 
$cos(t) * \vec{e}_{s+1} + sin(t) * \vec{p\prime}$.
Specifically,
$t_2 \mapsto (\vec{e}_{1}, \vec{e}_2, ..., \vec{e}_s, \vec{e}_{s+1}, ... , \vec{v\prime \prime}_k)$
and
$0 \mapsto (\vec{e}_1, \vec{e}_2, ..., \vec{e}_s, \vec{v\prime}_{s+1}, ... \vec{v\prime}_k)$.
The path must be continuous for it is made up of continuous input and operations. \\\\
Thus, by induction,
$SO(n)$
is path-connected for
$n \geq 2$. \\\\
Furthermore, by inspection, we saw that
$SO(1)$
is path-connected. Therefore,
$SO(n)$
for
$n \geq 1$
is path-connected.

\newpage
\section*{Problem 3}
Problem: Let $r > 0$. Consider the set
$$X_r := \{ (a_1, a_2, ... , a_d) \in (\mathbb{S}^1)^{\times r} \mid a_i = a_j \text{ only if } i = j \} \subset (\mathbb{S}^1)^{\times r} \subset (\mathbb{R}^2)^{\times r} = \mathbb{R}^{2r}$$
What is the cardinality of 
$ \pi _0 (X_r)$
? \\\\
Solution: We note that elements of
$X_r$
are sets of
$r$
distinct points on the unit circle. Given an arbitrary set of points
$x \in X_r$,
there is guaranteed to exist a homeomorphism (specifically a rotation) that takes
$x_1 \in x$
to the point 
$(1,0)$.
Now consider the complement of
$x \subset \mathbb{S}^1$.
This will be the disjointed circle
$\mathbb{S}^1 \setminus x$.
There now exists a homeomorphism from
$\mathbb{S}^1 \setminus x$
to the interval
$(0, 2 \pi)$. \\\\
The points that were initially in
$x$
are now distributed as holes across the interval
$(0, 2 \pi)$.
We call this set of holes
$Y_{r-1}$.
Since the complements of
$X_r$
and 
$Y_{r-1}$
are homeomorphic,
$X_r$
and
$Y_{r-1}$
must also be homeomorphic by Theorem 3.2 from the textbook (Equivalent subsets have equivalent complements). Furthermore, since
$X_r \cong Y_{r-1}$,
it must also be true that
$| \pi_0 (X_r) | = | \pi_0 (Y_{r-1}) |$.
Now,
$|  \pi_0 (Y_{r-1}) |$
is equal to the number of orderings of 
$r-1$ numbers on the real line. This is given by
$(r - 1)!$.
Thus,
$| \pi_0 (X_r) | = (r - 1)!$

\newpage
\section*{Problem 4}
Problem: Is there a continuous surjection
$f: \mathbb{S}^2 \times \mathbb{S}^1 \to O(3)$
? \\\\
Solution: We will prove this by contradiction. Assume there exists a continuous and surjective 
$f: \mathbb{S}^2 \times \mathbb{S}^1 \to O(3)$.
Then, the map
$\gamma : \pi_0(\mathbb{S}^2 \times \mathbb{S}^1) \to \pi_0 (O(3))$
must also be surjective. We now disprove this statement. \\\\
We first examine the cardinality of
$\mathbb{S}^2 \times \mathbb{S}^1$.
The following was proved in Homework 3 Problem 3:
$\mathbb{S}^n$
is path connected if 
$n \geq 1$.
This implies that both
$\mathbb{S}^2$
and $\mathbb{S}^1$
are path connected. Furthermore, it was proved in Homework 3 Problem 4 that
$X \times Y$
is path-connected iff
$X$
and
$Y$
are each path-connected.
Thus,
$\mathbb{S}^2 \times \mathbb{S}^1$
is path connected. This implies that
$| \pi_0 (\mathbb{S}^2 \times \mathbb{S}^1) | = 1$. \\\\
We now turn to the cardinality of
$O(3)$.
The set of matrices that compose
$O(3)$
must orthonormal. We now choose two specific orthonormal matrices and show that they are in different path components. Take a matrix 
$A$
and the matrix
$A^\prime$
where the columns of
$A^\prime$
are equal to the columns of
$A$
with the exception of the first column. Set
$\vec{a\prime}_1 = -1 * \vec{a}_1$.
The matrices
$A$
and
$A^\prime$
are identical with the exception that one pair of their vectors are scaled by -1. We can then begin rotating
$\vec{a\prime}_1$
to
$\vec{a}$
while staying within the same path component. However, we cannot completely rotate
$\vec{a\prime}_1$
since, to reach
$\vec{a}_1$,
$\exists \vec{a}_i, \vec{a}_j \text{ such that } \vec{a\prime}_1 \in span\{\vec{a}_i, \vec{a}_j \}$
at some point. This means that, at some point in the rotation,
$A^\prime$
will not be orthonormal. This implies that
$\pi_0 O(3) > 1$. \\\\
Since
$1 \nleq 1$,
know that the number of path components of
$\mathbb{S}^2 \times \mathbb{S}^1$
is different than the number of path components of
$O(3)$. Thus, a contradiction is found and there does not exist a continuous surjection
$f: \mathbb{S}^2 \times \mathbb{S}^1 \to O(3)$.

\end{document}